\documentclass[10pts, a4paper]{article}
\usepackage[utf8]{inputenc}
\usepackage{algorithm,algorithmicx,algpseudocode}
\usepackage{amssymb}
\usepackage{latexsym}
\usepackage{braket}
\usepackage{cryptocode}
\usepackage[english]{babel}
\setlength{\parindent}{4 em}
\setlength{\parskip}{1 em}
\renewcommand{\baselinestretch}{1.0}
\usepackage[margin=1.0in]{geometry}
\usepackage{float}
\floatstyle{boxed} 
\usepackage{amsmath}
\usepackage{caption}
\usepackage[superscript, biblabel, nomove]{cite}
\usepackage{mathtools}
\title{ \begin{large}
\textbf{Flow Construction}
\end{large} 
}
%\author{}
%document
\date{}
\begin{document}
\maketitle
Interpret Flow notations MBQC as follows. There are four kinds of outputs:
\begin{itemize}
\item E [i,j]: entangle qubits i and j using Controlled Phase. 
\item M i $\phi$ [m] [n]: Measurement angle for qubit i is $\alpha=(-1)^{\textnormal{(m mod 2)}}\phi+n\pi$
\item X j i: Operation of conditional X on qubit j depending on measurement outcome of qubit i
\item Z j i: Operation of conditional Z on qubit j depending on measurement outcome of qubit i
\item X's and Z's can be accommodated in mesurement angle as such: $M_j^\alpha X_j^i Z_j^k$: Measurement angle of qubit $j=(-1)^{s_i} \alpha+s_k\pi$, where $s_i$, $s_k$ are the measurement outcomes of qubits i and k respectively.
\end{itemize}
\end{document}




